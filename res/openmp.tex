\subsection{Wie funktioniert das?}
\begin{frame}
	\frametitle{\secname}
	\framesubtitle{\subsecname}
	
	\begin{itemize}
		\item Für C/C++, Fortran
			\note[item]{Aktuell bei gcc-5.* OpenMP 4.0}
		\item ähnlich mächtig zu PThreads, std::thread
			\note[item]{}
		\item Präprozessormakros (\#pragma)
			\note[item]{Steuerung, Angabe wie parallelisiert werden soll, Definition von Compileroptionen, wenn nicht ausgewertet werden kann, einfach ignoriert}
		\item integriert in Compiler
			\note[item]{Compiler muss unterstützen, aber alle gängigen}
	\end{itemize}
\end{frame}

\subsection{Wie sieht das aus?}
\begin{frame}
	\frametitle{\secname}
	\framesubtitle{\subsecname}
		
	\lstinputlisting{openmp.cpp}
		\note{kleine Beispielanwendung}
		\note[item]{$\{2\}$ OpenMP Header}
		\note[item]{$\{7\}$ Magic, man kann die Anzahl der Threads angeben}
		\note[item]{$\{10\}$ auf Threadnummer zugreifen}
\end{frame}

\begin{frame}
	\frametitle{\secname}
	\framesubtitle{\subsecname}
	
	\lstinputlisting{openmp.out}
		\note{Output bei 4 Threads, mehrere Sachen sehen:}
			\note[item]{mehrere schreiben gleichzeitig, durcheinander, leerzeilen}
			\note[item]{nicht in fester Reihenfolge}
			\note[item]{}
			\note[item]{Wie siehts mit EMB2 aus}
\end{frame}
